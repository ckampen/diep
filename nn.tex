\documentclass[11pt,a4paper]{article}
\usepackage{mathtools}
\usepackage{listings}
\usepackage{booktabs}
\usepackage{amsfonts}

\begin{document}
\title{Deep Neural Nets}

\author{Casper van Kampen}
\date{}

\maketitle
\newpage
\tableofcontents
\newpage

% \begin{abstract}
% This is the abstract section\ldots
% \end{abstract}


\section{Softmax classifier}

Squash all elements in a vector $(\mathrm{z})$ to a value below 1. The sum of this vector will be 1.
\\[11pt]
$\sigma{(\mathrm{z})}_j = \frac{\mathrm{e}^{z_j}}{\sum_{i=1}^{k} \mathrm{e}^{z_k}}$
\\
\lstinputlisting[language=Python, frame=single]
{softmax.py}

\newpage
\section{Activation functions}

Functions that decide if a neuron activates or not.

\subsection{ReLU}

Rectified Linear Unit
\\[11pt]
$f(x) = \max(0,x)$
\\
\lstinputlisting[language=Python, frame=single]
{relu.py}

Or with numpy:

\lstinputlisting[language=Python, frame=single]
{relu_np.py}

\subsection{Sigmoid}

$\sigma(x) = \frac{1}{1 + \mathrm{e}^{-x}}$
\\
\lstinputlisting[language=Python, frame=single]
{sigmoid.py}
% \\
% $y = \frac{1}{1 + \mathrm{e}^{-x}}$
% \\
derivative:\\[11pt]
$\frac{d\sigma(x)}{dx} = \frac{\mathrm{e}^{x}}{{(1 + \mathrm{e}^{x})}^2}$
\\[11pt]
$\frac{d\sigma(x)}{dx} = \sigma(x) (1 - \sigma(x))$
Research this!

\subsection{tanh}

$\frac{1 + \mathrm{e}^{-2x}}
	  {1 - \mathrm{e}^{-2x}}$
\\[11pt]
Or using the sigmoid function:
\\[11pt]
$2\sigma(2x) -1$
\\[11pt]
thanh is a scaled version of sigmoid.
\\[11pt]
\lstinputlisting[language=Python, frame=single]
{tanh.py}

\newpage
\section{Cost functions}

\subsection{Cross Entropy}

$\mathrm{H}(p,q) = - \displaystyle\sum_{x} p_x \log{q_x}$\\[11pt]
Use Cross entropy cost function so 2 networks with the same result but with different cerntenty will not have the same error.\\
e.g.\\
in a 3 classes classification problem consider the followin 2 hypotheses (result from a softmax layer):\\
\begin{table}[h]
\centering
\caption{Net 1}
\label{table:name}
\begin{tabular}{lcr}
\toprule
\textbf{H} & \textbf{Y} & \textbf{correct}\\
\midrule
0.3 0.3 0.4 & 0 0 1 & 1\\
0.3 0.4 0.3 & 0 1 0 & 1\\
0.1 0.2 0.7 & 1 0 0 & 0\\
\bottomrule
\end{tabular}
\end{table}
\begin{table}[h]
\centering
\caption{Net 2}
\label{table:name}
\begin{tabular}{lcr}
\toprule
\textbf{H} & \textbf{Y} & \textbf{correct}\\
\midrule
0.1 0.2 0.7 & 0 0 1 & 1\\
0.1 0.7 0.2 & 0 1 0 & 1\\
0.3 0.4 0.3 & 1 0 0 & 0\\
\bottomrule
\end{tabular}
\end{table}
\\[11pt]
correctness Net 1 = $(1 + 1 + 0) / 3 = \frac{2}{3}$\\
correctness Net 2 = $(1 + 1 + 0) / 3 = \frac{2}{3}$\\
the error rate = $ 1 - \frac{2}{3} = \frac{1}{3}$ in both cases.\\[11pt]
both nets have the same result but net 2 is much more certain in ist prediction.\
it is statistically much closer to te answer.\
we use cross entropy so that this difference translates to the error of the net
\\[11pt]
Calulating the cross entropy for Net 1.\\
\begin{math}
  net 1_1 = -(0 * \log(0.3)) + (0 * \log(0.3)) + (1 * \log(0.4)) = \log(0.4) = 0.916\ldots\\[11pt]
  net 1_2 = -(0 * \log(0.3)) + (1 * \log(0.4)) + (0 * \log(0.3)) = \log(0.4) = 0.916\ldots\\[11pt]
net 1_3 = -(1 * \log(0.1)) + (0 * \log(0.2)) + (0 * \log(0.7)) = \log(0.1) = 2.302\ldots\\[11pt]
net 1 = (-\log(0.4) + -\log(0.4) + -\log(0.1)) /3 = 1.378\ldots\\[11pt]
\end{math}
And then Net 2.\\
\begin{math}
  \mathrm{H}(net2_{1}) = -(0 * \log(0.1)) + (0 * \log(0.2)) + (1 * \log(0.7))\\ = -\log(0.7)\\ = 0.356\ldots\\[11pt]
  \mathrm{H}(net2_{2}) = -(0 * \log(0.1)) + (1 * \log(0.7)) + (0 * \log(0.2))\\ = -\log(0.7)\\ = 0.356\ldots\\[11pt]
  \mathrm{H}(net2_{3}) = -(1 * \log(0.3)) + (0 * \log(0.4)) + (0 * \log(0.3))\\ = -\log(0.3)\\ = 1.203\ldots\\[11pt]
  net_{2} = (-\log(0.7) + -\log(0.7) + -\log(0.3)) /3 = 0.639\ldots\\[11pt]
\end{math}
We see that Net 1 has a higher error than Net 3 over the 3 samples given.
\lstinputlisting[language=Python, frame=single]
{cross_entropy.py}
\newpage
\section{Gradient descent}
\subsection{Back propagation}
\newpage
\section{Neurons}

\subsection{Feed Forward}
Neuron architecture\\[11pt]
$X \to N (W X + b) \to Activation (ReLU, \sigma, tanh) \to [0, y]$

Network layer is given by the formula:\\[11pt]
$W X + b$\\[11pt]

\subsection{Matmul}


Multiplication of two matrices:

\begin{math}
  Y = WX =
  \begin{pmatrix}
    1 & 0 & 0\\
    -2 & 1 & 0\\
    0 & 0 & 1
  \end{pmatrix}
  \begin{pmatrix}
    2 & 4 & -2\\
    4 & 9 & -3\\
    -2 & -3 & 7
  \end{pmatrix}
  =
  \begin{pmatrix}
    2 & 4 & -2\\
    0 & 1 & 1\\
    -2 & -3 & 7
  \end{pmatrix}
\end{math}

$
\vec{\mathbf{Y_1}} =\\
( 1 \times 2) + (0 \times 4) + (0 \times -2) = 2,\\
( 1 \times 4) + (0 \times 9) + (0 \times -3) = 4,\\
( 1 \times -2) + (0 \times -3) + (0 \times 7) = -2
$

$
\vec{\mathbf{Y_2}} =\\
(-2 \times 2) + (1 \times 4) + (0 \times -2) = 0,\\
(-2 \times 4) + (1 \times 9) + (0 \times -3) = 1,\\
(-2 \times -2) + (1 \times 3) + (0 \times 7) = 1
$

$
\vec{\mathbf{Y_3}} =\\
( 0 \times 2) + (0 \times 4) + (1 \times -2) = -2,\\
( 0 \times 4) + (0 \times 9) + (1 \times -3) = -3,\\
( 0 \times -2) + (0 \times -3) + (1 \times 7) = 7
$
\lstinputlisting[language=Python, frame=single]
{matmul.py}

\subsection{Dot product}

Linear network layer containing $\mathrm{m}$ neurons and $\mathrm{m} \times \mathrm{n}$ inputs.\\[11pt]
\[
\begin{pmatrix}
  w_{1} \\ % & w_{1,2} & \cdots & w_{1,n} \\
  w_{2} \\ % & w_{2,2} & \cdots & w_{2,n} \\
  \vdots \\ %i & \vdots & \ddots & \vdots \\
  w_{m} % & w_{m,2} & \cdots & w_{m,n}
\end{pmatrix}
\bullet
\begin{pmatrix}
  x_{1,1} & x_{1,2} & \cdots & x_{1,n} \\
  x_{2,1} & x_{2,2} & \cdots & x_{2,n} \\
  \vdots & \vdots & \ddots & \vdots \\
  x_{m,1} & x_{m,2} & \cdots & x_{m,n}
\end{pmatrix}
+
\begin{pmatrix}
  b_{1} \\ % & w_{1,2} & \cdots & w_{1,n} \\
  b_{2} \\ % & w_{2,2} & \cdots & w_{2,n} \\
  \vdots \\ %i & \vdots & \ddots & \vdots \\
  b_{m} % & w_{m,2} & \cdots & w_{m,n}
\end{pmatrix}
=
\begin{pmatrix}
  w_{1}x_{1,1}+w_{1}x_{1,2}\cdots + w_{1}x_{1,n}+b_{1}\\
  w_{2}x_{2,1}+w_{2}x_{2,2}\cdots + w_{2}x_{2,n}+b_{2}\\
  \vdots \\
  w_{m}x_{m,1}+w_{m}x_{m,2}\cdots + w_{m}x_{m,n}+b_{m}
\end{pmatrix}
\]

Simple Numpy function.

\lstinputlisting[language=Python, frame=single]
{linear.py}

\newpage
\section{Tensorflow}

\subsection{Load Data}

First, we will need to load the training and test data.
We will use the MNIST dataset for this example to classify images of digits.
Tensorflow provides functions to deal with MNIST data so we will use this to
load the data.

\lstinputlisting[language=Python, basicstyle=\tiny, frame=single]
{tf/data.py}

\newpage
\subsection{Network layout}

We define the layout of the net in terms of layers.
Tensorflow wlil create a `computation graph' in memory that wil be executed
in the training stage. There is no computation done in this stage.

\lstinputlisting[language=Python, basicstyle=\tiny, frame=single]
{tf/neural_net.py}

\subsubsection{logits}
The logits function is the inverse of the logistic function.
Logit is a function that maps probabilities ([0, 1]) to $\mathbb{R} ((-\infty, \infty))$

$\mathrm{L} = \ln{\frac{p}{1 - p}}$\\
$\mathrm{p} = \frac{1}{1 + \mathrm{e}^{-l}}$


\newpage
\subsection{Learning}
\lstinputlisting[language=Python, basicstyle=\tiny, frame=single]
{tf/learning.py}

\end{document}
